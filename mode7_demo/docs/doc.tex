Writing a Graphics Demo for the Apple II

By DEATER
AKA Vincent M. Weaver

I was writing a game for the Apple II and realized I had come up with
some clever Super-Nintendo (SNES) style graphics routines that were just
crying to be turned into a demo-scene style demo.

The Apple II was the first computer I had access too, and I grew up in an odd
neighborhood where it was all Apples and not a Commodore to be seen.
My family long ago got rid of our machine, but I rescued an Apple IIe platinum
from the dumpster one day and have dragged it from state to state ever since.

I find 6502 assembly to be oddly theraputic, and will code in it when other
projects become too stressful.  Especially when Linux up and hangs on me
because firefox tried to do something stupid in javascript.  I then pine for
the days when you could do something useful in 64k of RAM, and not have your
machine fall over because somehow 4GB is not enough.

Setup:

I do my development on Linux, using the nano text editor.  I use the
ca65 assembler from the cc65 project, which I find to be a reasonable
tool although most "real" Apple II programmers look down on it for some
reason.
I cross-compile on x86 Linux, construct Apple DOS33 disk images using
some tools I've written, and then do most testing in an emulator.
(These days usually AppleWin under the wine emulator, or else MESS/MAME
which has cleaner sound output).  Once things work then I'll stick things
on a USB stick and transfer to the CFFA3000 disk emulator installed in
the actual Apple II.

Related Work:
See anything by the group FrenchTouch, whose Apple II demos outclass
mine by a lot.


Background:

The Apple II was the first computer I programmed on, lo many years ago.
Mostly in Applesoft BASIC (which ended up being the only Microsoft product
I ever liked) but I was starting to get into assembly language about the
time my family got a 386 system.

I've revisited over the years, with some 6502 programming to show I could.
My skills were not that great, I had one of my size-optimization projects
crowd re-optimized.  For a while I had a side-gig re-optimizing modern games
in BASIC, before getting sidetracked into going full in on 6502 assembly
again.

Introduced in 1977.
The Apple II runs at 1.XX check Megahertz.  6502, which can easily
address 64 kB of RAM (more with bank switching).  Shipped with as little
as 4kB of RAM.  Three registers, (A,X,Y) but a large ``zero page'' which
gives you register-like actions on the first 256 bytes of RAM.

DOS3.3 operating system with 140k floppies.  Amazing programming by Wozniak,
allowing all kinds of floppy protection shenanigans (cite 4am, previous
article).

Graphics

The Apple II
	Low-res: 40x48 NTSC 15-color 
	Hi-res: X x Y 6-colors
	Both, optional text on bottom.

Did have HW double buffering, but no graphics chars, no reprogrammable
font, no sprite hardware, no retrace interrupt.

Could do amazing things by cycle counting, reading the floating bus,
 and racing the beam and flipping 
graphics modes on the fly, but that's an entire different demo, some other day.

Later models added double low-res (80x48) and double hi-res (x y in
NTSC 15 color) but didn't appear until 198x, and only on later IIe, IIc
models.

Apple also came out with the IIgs which arguably was much more advanced
and cheaper than the Mac, but Apple cancelled the II line much to the
sadness of the users (Apple II forever).

Sound:
	1-bit beeper sa they say, though some people managed to
do sme impressive things (link to ED).  1981 Mockingboard
card, just two AY-3-8910 chips linked to the bus with 6522 chips.


Draw memory map.  $ means hexadecimal traditionally.


Title
	My poor attempt at blocky graphics.  Not the best pixel artist,
	just as I was a mediocre BBS/ANSI artist back in the day.

	Fading, a bit of a hack, no pallette hardware.  Just have
	a third copy of background in RAM and then memcopy with a lookup
	table for the three levels of fade.

	Scrolling text, tearing, if was more clever would use the
	refresh interrupt available on newer machines or do cycle counting.

Bouncing sphere

3D

Starfield

Rasterbars

Mockingboard Sound
